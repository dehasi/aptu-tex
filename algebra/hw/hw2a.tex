\documentclass[a4paper,12pt]{article}

\usepackage[T2A]{fontenc} 
\usepackage[utf8]{inputenc}
\usepackage[english,russian]{babel}
\usepackage{listings}
\usepackage[dvips]{graphicx}
\usepackage{indentfirst}
\usepackage{color}
\usepackage{hyperref}
\usepackage{amsmath}
\usepackage{amssymb}
\usepackage{geometry}
\geometry{left=2cm}
\geometry{right=2cm}
\geometry{top=2cm}
\geometry{bottom=2cm}

\graphicspath{{images/}}

\begin{document}
\sloppy

\lstset{
	basicstyle=\small,
	stringstyle=\ttfamily,
	showstringspaces=false,
	columns=fixed,
	breaklines=true,
	numbers=right,
	numberstyle=\tiny
}

\newtheorem{Def}{Определение}[section]
\newtheorem{Th}{Теорема}
\newtheorem{Lem}[Th]{Лемма}
\newenvironment{Proof}
	{\par\noindent{\bf Доказательство.}}
	{\hfill$\scriptstyle\blacksquare$}
\newenvironment{Solution}
	{\par\noindent{\bf Решение.}}
	{\hfill$\scriptstyle\blacksquare$}


\begin{flushright}
	Кринкин М. Ю. группа 504 (SE)
\end{flushright}

\section{Домашнее задание 2}

\paragraph{3.6} Докажите, что $G$ - циклическая подгруппа порядка $p^n$, где $p$ - простое число и $n \in \mathbb{N}$, если и только если существует собственная подгруппа группы $G$, содержащая все собственные подгруппы группы $G$.

\begin{Proof}
Как я понимаю полагается все-таки, что $n \in \mathbb{N} \setminus \{1\}$, так как при $n = 1$ получаем просто группу простого порядка, но она не имеет собственных подгрупп.

Теперь докажем, что из того, что циклическая группа имеет примарный порядок, то в ней содержится собственная подгруппа содержащая в себе все собственные подгруппы.

Если $G$ циклическая, то существует $x : <x> = G$, собственную циклическую подгруппу в $G$ могут образовывать только элементы с порядком $p^k$, где $k \in \overline{2,n-1}$, так как кроме них в группе существует лишь элементы порядка 0 или порядка $p^n$, которые образуют несобственные подгруппы.

Рассмотрим элемент $y = x^p$ образующий максимальную собственную подгруппу $H$. Действительно, если расписать:
\[
	\underbrace{\overbrace{x ... x}^p ... \overbrace{x ... x}^p}_{p^n} = \underbrace{y ... y}_{p^{n-1}}
\]
но $p^{n-1}$ максимальный возможный порядок собственной подгруппы. Более того любой элемент группы $G$ представим как $x^m$, но чтобы он порождал подгруппу необходимо, чтобы $p^n \vdots m$, т. к. $p$ - простое, то $m = p^k$, а любой $x^{\left(p^k\right)} = \left(x^p\right)^{\left(p^{k-1}\right)}$ т. е. любой элемент порождающий собственную подгруппу входит в подгруппу $H$.

Теперь покажем, что из существования собственной подгруппы $H < G$, такой, что все собственные подгруппы содержаться в $H$. Возьмем элемент $z \in G \setminus H$, пусть он имеет порядок отличный от порядка группы, тогда он порождает собственную подгруппу в $G$, но по условию эта подгруппа содержится в $H$, следовательно и $z \in H$, что противоречит условию $z \in G \setminus H$, следовательно, порядок $z$ совпадает с порядком $G$, а значит $G$ - циклическая.
\end{Proof}

\end{document}
