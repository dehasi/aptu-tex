\chapter{Орбиты, стабилизаторы и Ко}

После долгого перерыва, наш неформальный конспект наконец-то возобновляется.
И блин прикольно, он начинается с комбинаторики...

\section{В предыдущих сериях}

\begin{Def}
$G$-группа, $X$-множество, тогда $X$ - $G$-множество, если задана операция:
\[
	\begin{split}
		& \cdot : G \times X \rightarrow X \\
		& \left(g, x\right) \mapsto gx
	\end{split}
\]
или что тоже самое
\[
	\begin{split}
		& \phi : G \rightarrow S\left(X\right) \\
		& g \mapsto \left(\phi_g : x \mapsto gx\right)
	\end{split}
\]
\end{Def}

\begin{Def}
$G$-группа, $X$ - $G$-множество, тогда орбитой элемента $x \in X$ называется множество $Gx = \lbrace gx | g \in G\rbrace \subseteq X$.
\end{Def}

Теперь вброс, отношение $x,y \in X, x \equiv_G y \Leftrightarrow \exists g \in G \left[y = gx\right]$ - отношение эквивалентности, т. е. два элемента эквивалентны, если принадлежат одной орбите.

Докажем это:

\begin{enumerate}
\item Симметричность: $x \equiv y \Rightarrow y = gx \Rightarrow x = g^{-1}y \Rightarrow y \equiv y$

\item Транзитивность: $x \equiv y \equiv z \Rightarrow x = gy = h\left(gy\right) \Rightarrow x = \left(hg\right)z \Rightarrow x \equiv z$

\item Рефлексивность: $x = 1 x \Rightarrow x \equiv x$.
\end{enumerate}

\begin{Def}
$X / G$ - множество орбит в $X$ под действием группы $G$, или $\lbrace Gx | x \in X\rbrace$

или

$X = \sqcup_{o \in X / G} o$ - множество $X$ - дизъюнктивное объединение орбит.
\end{Def}

\begin{Def}
Пусть $G$ - группа, $X$ - $G$-множество, $x \in X$

$St_G\left(x\right) = \lbrace g \in G | gx = x\rbrace \le G$ - стабилизатор точки $x \in X$ - множество элементов группы, оставляющих ее на месте.
\end{Def}

\begin{Def}
$G$ - группа, $X$ - $G$-множество, тогда

$Fix_X\left(g\right) = \lbrace x \in X | gx = x\rbrace \subseteq X$ - множество неподвижных точек элемента $g \in G$.
\end{Def}

\paragraph{Пример: }
Пусть $X$ и $Y$ - множества. Рассмотрим отображение $S\left(X\right) \times Y^X \rightarrow Y^X$, действующее по правилу $\left(u, f\right) \mapsto f \circ u^{-1}$, где $u \in S\left(X\right)$ и $f \in Y^X$, оно задает действие группы $S\left(X\right)$ на множестве всех отображений из $X$ в $Y$.

Рассмотрим как выглядит орбита элемента множества $Y^X$. Пусть есть некоторая функция $f \in Y^X$, которая отображает элементы из $X$ в элементы из $Y$, тогда $f \circ u^{-1}$ функция, которая сопоставляет элементу $x \in X$ элемент $f\left(u^{-1}\left(x\right)\right)$. Так как $u \in S\left(X\right)$ - задает все биекции, то действие на $f$ элементом $u$ меняет прообраз функции, но не меняет ее образ, так как не добавляет новых элементов и не удаляет старых из $X$.

Теперь рассмотрим стабилизатор некоторого отображения $f$. Чтобы элемент $u^{-1}$ не изменил отображение $f$ необходимо, чтобы $f\left(x \in X\right) = f\left(u^{-1}\left(x\right)\right)$. Т. е. стабилизатор образуют все биекции для которых $f\left(x\right) = f\left(u^{-1}\left(x\right)\right)$, в конечномерном случае, это будут все перестановки такие, что внутри одного цикла находятся только элементы с одинаковым образом.

А множеством неподвижных точек для некоторой перестановки $u$, будет содержать все отображения $f$ такие, что $f\left(x\right) = f\left(u^{-1}\left(x\right)\right)$, в конечномерном случае, это будут все отображения, постоянные внутри цикла перестановки.

Теперь возьмем следующее $G$-множество и операцию:

\begin{enumerate}
\item $G\times G \rightarrow G$

\item $\left(g_1, g_2\right) \mapsto g_1g_2$
\end{enumerate}

или что тоже самое

\begin{enumerate}
\item $G \rightarrow S\left(G\right)$

\item $g_1 \mapsto \left(g_2 \mapsto g_1g_2\right)$
\end{enumerate}

действие - умножение слева.

\begin{Th}[Кэли]
Отображение $\rho : G \rightarrow S\left(G\right)$ - мономорфизм
\end{Th}

\begin{Proof}
Требуется доказать справедливость $\rho_{g_1} = \rho_{g_2} \Rightarrow g_1 = g_2$. Для этого воспользуемся тем, что $\rho_{g_1} \left(1\right) = g_1$ и $\rho_{g_2} \left(1\right) = g_2$, откуда вместе с $\rho_{g_1} = \rho_{g_2}$ необходимо получаем $g_1 = g_2$.
\end{Proof}

\begin{Def}
$G$-группа, $X,Y$ - $G$-множества, тогда $f : X \rightarrow Y$ - морфизм, если
\[
	\forall g \in G, x \in X \left[f\left(g x\right) = g f\left(x\right)\right]
\]
\end{Def}

\begin{Def}
$f$ - изоморфизм, если он является биективным морфизмом.
\end{Def}

\begin{Def}
$X \cong Y$ - равносильно тому, что существует изоморфизм $f : X \rightarrow Y$
\end{Def}

\begin{Def}
$X$ - однородное $G$-множество, если оно покрывается всего одной орбитой, т. е.
\[
	\exists x \in X \left[X = Gx\right]
\]
\end{Def}

\begin{Th}
\begin{enumerate}
\item если $H \le G$, то $G / H$ - однородное $G$-множество при действии сдвигами

\item если $X$ - однородное, то $\forall x \in X \left[X \cong G / St_G\left(x\right)\right]$
\end{enumerate}
\end{Th}

\begin{Proof}
Первое очевидно.

Второе, пусть $X = Gx$ для некоторого $x \in X$, построим отображение:
\[
	\begin{split}
		& G / St_G\left(x\right) \rightarrow X \\
		& g St_G\left(x\right) \mapsto gx
	\end{split}
\]
проверим инъективность: $g_1 x = g_2 x \Leftrightarrow g_2^{-1} g_1 x = x \Leftrightarrow g_2^{-1} g_1 \in St_G\left(x\right) \Leftrightarrow g_1 St_G\left(x\right) = g_2 St_G\left(x\right)$

Сюрекция же вытекает просто из определения однородности.

Осталось показать, что отображение является морфизмом:

$f\left(g'\left(gSt_G\left(x\right)\right)\right) = f\left(g'gSt_G\left(x\right)\right) = \left(g'g\right)x = g'\left(gx\right) = g'f\left(gSt_G\left(x\right)\right)$
\end{Proof}

\begin{Th}[Лемма Бернсайда]
$\left|G\right| < \infty$, $X$ - $G$-множество, тогда
\[
	\frac{1}{\left|G\right|}\sum_{g \in G} \left|Fix_X\left(g\right)\right| = \left|X/G\right|
\]
Или средний размер множества неподвижных точек равен число орбит $G$-множества.
\end{Th}

\begin{Proof}
\[
	\begin{split}
		& \frac{1}{\left|G\right|} \sum_{g \in G} \left|\lbrace x | gx = x\rbrace\right| = \frac{1}{\left|G\right|} \left|\lbrace \left(g,x\right) | gx = x \rbrace\right| = \\
		& = \frac{1}{\left|G\right|} \sum_{x \in X} \left|\lbrace g | gx = x\rbrace\right|
	\end{split}
\]
а так как $Gx \cong G / St_G\left(x\right)$, то отношение мощности группы к мощности орбиты некоторого элемента $x$ равно мощности его стабилизатора, следовательно:
\[
	\begin{split}
		& \frac{1}{\left|G\right|} \sum_{x \in X} \left|\lbrace g | gx = x\rbrace\right| = \frac{1}{\left|G\right|} \sum_{x \in X} \frac{\left|G\right|}{\left|Gx\right|} = \\
		& = \sum_{x \in X} \frac{1}{\left|Gx\right|} = \sum_{o \in X/G} \sum_{x \in o} \frac{1}{\left|o\right|} = \left|X/G\right|
	\end{split}
\]
\end{Proof}

Тут бы уместно привести приер задачки, которая решается этой леммой, ну вот вам пример:

Пусть есть квадрат, требуется нати все существенно различные окраски вершин квадрата в два цвета. Существенно различными окрасками называем окраски, не получающиеся друг из друга поворотами на 90, 180 или 270 градусов, или симметриями относительно вертикальной или горизонтальной оси, а также относительно диагоналей квадрата.

Пффууух... Итак рассмотрим сначала действие симметричного отражение относительно вертикали, очевидно, что чтобы окраска перешла сама в себя, то необходимо, чтобы верхние углы квадрата имели один цвет, и нижные углы квадрата имели один цвет, очевидно способов такой раскраски ровно 4, аналогично и относительно горизонтальной симметрии.

Теперь рассмотрим диагональную симметрию, очевидно, что неважно как раскрашени вершины диагонали, относительно которой проивзодится симметрия, в то время как оставшиеся две вершины должны быть покрашены в один цвет, число способов покрасить так квадрат 8 на каждую диагональ.

Наконец мои любимые повороты. Сначала поворот на 90, тут все нетрудно, чтобы при повороте раксраска не изменилась необходимо, чтобы все вершины были покрашены в один цвет, аналогино и для поворота на 270, т. е. по 2 окраски на каждый поврот. Поворот на 180 уже менее привередлив, необходимо, чтобы вершины одной диагонали были окрашены в один цвет, а это 4 окраски.

Все? А вот и нет, чтобы преобразования образовывали группу, необходимо иметь единичное преобразование - непреобразование, которое оставляет на месте любую раскраску, т. е. относительно нее инвариантны все раскраски, а таковых $2^4 = 16$

Теперь забиваем в нашу формулу:
\[
	\frac{8}{4 + 4 + 8 + 8 + 2 + 2 + 4 + 16} = 6
\]

упс ошибочка вышла... старею... Надо к Д. Кнут сделать... если после моей смерти в этом конспекте будут найдены ошибки, считать их фичами)
\[
	\frac{1}{8} \cdot \left(4+4+8+8+2+2+4+16\right) = 6
\]

Решить такую задачу можно и без леммы Бернсайда, достаточно заметить, что в неразличимых относительно поворотов и симметрий способах раскраски количество вершин одного цвета совпадает, это замечание упрощает решение, но работает такой метод как правило только на задачах небольшой размерности, как в нашем случае.

\begin{Def}
$p \in \mathbb{P}$, $G$ - конечная $p$-группа, если $\left|G\right| = p^n$
\end{Def}

Теперь непонятная теоремка, но может пригодится

\begin{Th}
Пусть $G$ - конечная $p$-группа и действет на конечном множестве $X$, тогда
\[
	\left|X\right| = \text{кол-во одноэлементых орбит} \mod p
\]
\end{Th}

\begin{Def}
Известно, что $X$ покрывается непересекающимися орбитами. Кроме того, так как $X_i \cong G / St_G\left(x_i\right)$, где $X_i$ - орбита элемента $x_i$, получаем (очевидное следствие теоремы об однородных $G$-множествах), что из $\left|G\right| \vdots \left|X_i\right|$ при $\left|X_i\right| \ge 2$ $\left|X_i\right| \vdots p$, это значит, что:
\[
	\left|X\right| = \sum 1 + \sum p k
\]
где $k \in \mathbb{N}$, а $\sum 1$ - вклад одноэлементых орбит в мощность множества.

Отсюда очевидно получаем требуемое:
\[
	\left|X\right| = \sum 1 \mod p
\]
\end{Def}
