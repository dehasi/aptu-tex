\documentclass[a4paper,12pt]{article}

\usepackage[T2A]{fontenc} 
\usepackage[utf8]{inputenc}
\usepackage[english,russian]{babel}
\usepackage{listings}
\usepackage[dvips]{graphicx}
\usepackage{indentfirst}
\usepackage{color}
\usepackage{hyperref}
\usepackage{amsmath}
\usepackage{amssymb}
\usepackage{geometry}
\geometry{left=1.5cm}
\geometry{right=1.5cm}
\geometry{top=1cm}
\geometry{bottom=2cm}

\graphicspath{{images/}}

\begin{document}
\sloppy

\lstset{
	basicstyle=\small,
	stringstyle=\ttfamily,
	showstringspaces=false,
	columns=fixed,
	breaklines=true,
	numbers=right,
	numberstyle=\tiny
}

\newtheorem{Def}{Определение}[section]
\newtheorem{Th}{Теорема}
\newtheorem{Lem}[Th]{Лемма}
\newenvironment{Proof}
	{\par\noindent{\bf Доказательство.}}
	{\hfill$\scriptstyle\blacksquare$}
\newenvironment{Solution}
	{\par\noindent{\bf Решение.}}
	{\hfill$\scriptstyle\blacksquare$}


\begin{flushright}
	Кринкин М. Ю. группа 504 (SE)
\end{flushright}

\section{Домашнее задание 5}

\paragraph{19.} Если это возможно, укажите конкретные формулы $A$ и $B$, а также интерпритацию и оценку, при которых следующая формула ложна:

\begin{itemize}
\item (a) $A \lor \exists x B \equiv \exists x \left(A \lor B\right)$

\item (b) $A \supset \forall x B \equiv \forall x \left(A \supset B\right)$
\end{itemize}

\begin{Solution}
Рассмотрим пункт (a). Пусть $A = x > SS0$ и $B = x < SS0$, $x$ - переменная, тогда:
\[
	\left(x>SS0\right)\lor\exists x \left(x<SS0\right)\equiv\exists \left(\left(x>SS0\right)\lor\left(x<SS0\right)\right)
\]
Но левая часть зависит от оценки, допустим в некоторой оценке $\mu$ $\left|x\right|_{\mu} = 4$, тогда очевидно левая часть истинна, но првая часть ложна, так как $x$ в ней связана квантором существования, а значит все выражение в данной интерпритации ложно.

Теперь рассмотрим пункт (b). Пусть $A = B = x > S0$, тогда получаем:
\[
	\left(x > S0\right) \supset \forall \left(x > S0\right) \equiv \forall \left(\left(x > S0\right) \supset \left(x > S0\right)\right)
\]
Правая часть очевидно тождественно верна, но в левой части переменная $x$ не связана квантором всеобщности и значит выражение зависит от оценки, и при любой оценке сопоставляющей перемнной $x$ значение $> 1$ получаем, что левая часть ложна, а значит и все выражение ложно.
\end{Solution}

\paragraph{20.} Докажите, что конгруэнтные формулы равносильны.

\begin{Solution}
Идейно, утверждение следует из того, что конгруентные формулы отличаются связанными переменными. Но так как на связанные переменные налагют ограничения кванторы, то формулы должны быть равносильны.

Бескванторные формулы конгруентны если они совпадают, далее пусть в формуле есть всего один квантор.

Пусть $A$ и $B$ - конгруентны, будем рассматривать кванторы существования и всеобщности, а также подформулы, на которых эти кванторы действуют. Пусть $A' = \forall x A''$ - подформула формулы $A$, где $A''$ - безкванторная формула, но это означает, что в формуле $B$ на том же месте стоит подформула $B' = \forall y B''$, где $B''$ - бескванторная подформула, в которой каждому вхождению переменной $x$ в $A''$ соответствует вхождение переменной $y$ в $B''$. Пусть заданы интерпритация $M$ и оценка $v$, тогда если $\left|A'\right|_{M, v} = 1$, то для любого $\alpha \in D$ $\left|A''\right|_{M, v^{x}_{\alpha}} = 1$, а значит, что и для $\left|B''\right|_{M, v^{y}_{\alpha}} = 1$, если же $\exists \alpha \in D : \left|A''\right|_{M, v^{x}_{\alpha}} = 0$, то и $\left|B''\right|_{M, v^{y}_{\alpha}} = 0$.

Теперь пусть $A' = \exists x A''$ и $B' = \exists x B''$. Тогда, если $\left|A'\right|_{M,v} = 1$ значит, что $\exists \alpha \in D : \left|A''\right|_{M, v^{x}_{\alpha}} = 1$, но тогда $\left|B''\right|_{M, v^{y}_{\alpha}} = 1$, следовательно $\left|B'\right|_{M, v} = 1$, если же $\not\exists \alpha \in D : \left|A''\right|_{M, v^{x}_{\alpha}} = 1$, но это также значит, что и для переменной $y$ не существует такого $\alpha in D : \left|B''\right|_{M,v^{y}_{\alpha}} = 1$, а значит $\left|B'\right|_{M,v} = 0$.

Теперь пусть $A$ и $B$ конгруентны, и $A = \forall x A'$, $B = \forall y B'$. Если $\left|A\right|_{M, v} = 1$, значит $\forall \alpha \in D : \left|A'\right|_{M,v_{\alpha}^x} = 1$, а значит в этой интепритации и оценке, если $A'$ и $B'$ равносильны, то и $\left|B'\right|_{M,v_{\alpha}^y}$, если же $\left|A\right|_{M,v} = 0$, то следовательно существует $\alpha \in D : \left|A'\right|_{M,v_{\alpha}^x} = 0$, но тогда из равносильности $A'$ и $B'$ получаем $\left|B'\right|_{M,v_{\alpha}^y} = 0$, следовательно $\left|B\right|_{M, v} = 0$
\end{Solution}

\paragraph{21.} Докажите, что бескванторная предикатная формула общезначима тогда и только тогда, когда она является частным случаем пропозициональной тавтологии.

\begin{Solution}
Сопоставим множеству предикатных символов множество пропозициональных переменных и истинностных констант.

\begin{itemize}
\item Если $F\left(x_1, x_2, ..., x_n\right)$ - $n$-метный предикатный символ и $\forall \alpha_1, \alpha_2, ..., \alpha_n \in D : \left|F\left(\alpha_1, \alpha_2, ..., \alpha_n\right)\right|_M = 1$, то сопоставим ему истинностную константну $\mathbb{T}$

\item Если же при том же смысле $F$ получаем $\forall \alpha_1, \alpha_2, ..., \alpha_n \in D : \left|F\left(\alpha_1, \alpha_2, ...,\alpha_n\right)\right|_M = 0$, то соспоставим ему истинностную константну $\mathbb{F}$

\item Во всех остальных случаях сопоставим каждому предикатному символу $F$ пропозициональную переменную, причем всем функциональным символам $F_1, F_2, ... F_k$ таким что $F_1 \equiv F_2 \equiv ... \equiv F_k$ сопоставим одну и ту же пропозициональную переменную $x$. Функциональным символам, для которых это не выполняется сопоставляем различные пропозициональные перменные.
\end{itemize}

Тогда формула полученная из исходной таким отображением является пропозициональной формулой и является тождественно истинной, если является пропозициональной тавтологией, но по построению оценка новой формулы равна оценке исходной.
\end{Solution}

\paragraph{22.} Верно ли, что для любой формулы $A$ имеет место:
\begin{itemize}
\item (a) $A$ общезначима тогда и только тогда, когда $\neg \forall\negthickspace\forall\negthickspace\forall A$ невыполнима;

\item (b) $A$ общезначима тогда и только тогда, когда $\neg A$ невыполнима;

\item (с) $A$ невыполнима тогда и только тогда, когда $\forall\negthickspace\forall\negthickspace\forall A$?
\end{itemize} 

\begin{Solution}
Рассмотрим утверждение (a). Если формула $A$ общезначима, значит при любой интерпритации $M$ и любой оценке $v$ $\left|A\right|_{M,v} = 1$, пусть $x$ - не связанная квантором перемнная в $A$, тогда из предыдущего утверждения получаем, что $\left|\forall x A\right|_{M, v} = 1$, повторяем подобное рассужедение для каждого параметра $A$ получаем универсальное замыкание, интерпритация которого равна 1, а интерпритация отрицания равна 0, а значит отрицание универсального замыкания невыполнимо.

Теперь пусть отрицание универсального замыкания невыполнимо, тогда универсальное замыкание выполнимо. Из того, что $\forall x_1 \forall x_2 ... \forall x_n A$ - выполнима, следует, что для любой интерпритации и оценки выполнима формула $\forall x_2 ... \forall x_n A$, продолжая также по одному устранять кванторы всеобщности, навешанные на параметры $A$, получаем что исходная орфмула также выполнима.

Рассмотрим утверждение (b). Оценка формулы $\left|\neg A\right|_{M, v} = 0$, если и только если $\left|A\right|_{M, v} = 1$, но для любой интерпритации и оценки $\left|A\right|_{M, v} = 1$, а значит и $\left|\neg A\right|_{M,v} = 0$, верно и обратное.

Рассмотрим утверждение (c). Данное утверждение уже не верно. Из невыполнимости $A$ следует, что при любых оценке и интерпритации $\left|A\right| = 0$, а значит навешивая квантор всеобщности на параметры не меняем невыполнимости формулы. Но в обратную сторону это не верно, так как интуитивно снятие квантора всеобщности с перемнной - ослабление условия. Пусть некоторой переменной $x$ связанной квантором в $\forall\negthickspace\forall\negthickspace\forall A = \forall\negthickspace\forall\negthickspace\forall \forall x A$ и являющаяся параметром в $A$ в некоторой оценке $v$ сопоставлется $\alpha$ такое что $\left|\forall\negthickspace\forall\negthickspace\forall A\right|_{M, v^{x}_{\alpha}} = 0$, но при этом для того, чтобы формула $A$ была выполнима достаточно существование $\beta$ такое, что $\left|\forall\negthickspace\forall\negthickspace\forall A\right|_{M, v^{x}_{\beta}}=1$
\end{Solution}

\end{document}
