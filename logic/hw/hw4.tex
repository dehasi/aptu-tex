\documentclass[a4paper,12pt]{article}

\usepackage[T2A]{fontenc} 
\usepackage[utf8]{inputenc}
\usepackage[english,russian]{babel}
\usepackage{listings}
\usepackage[dvips]{graphicx}
\usepackage{indentfirst}
\usepackage{color}
\usepackage{hyperref}
\usepackage{amsmath}
\usepackage{amssymb}
\usepackage{geometry}
\geometry{left=1.5cm}
\geometry{right=1.5cm}
\geometry{top=1cm}
\geometry{bottom=2cm}

\graphicspath{{images/}}

\begin{document}
\sloppy

\lstset{
	basicstyle=\small,
	stringstyle=\ttfamily,
	showstringspaces=false,
	columns=fixed,
	breaklines=true,
	numbers=right,
	numberstyle=\tiny
}

\newtheorem{Def}{Определение}[section]
\newtheorem{Th}{Теорема}
\newtheorem{Lem}[Th]{Лемма}
\newenvironment{Proof}
	{\par\noindent{\bf Доказательство.}}
	{\hfill$\scriptstyle\blacksquare$}
\newenvironment{Solution}
	{\par\noindent{\bf Решение.}}
	{\hfill$\scriptstyle\blacksquare$}


\begin{flushright}
	Кринкин М. Ю. группа 504 (SE)
\end{flushright}

\section{Домашнее задание 4}

\paragraph{Задание 15.} Запишите формулу языка элементарной арифметики, выражающую в стандартной интерпритации этого языка следующее:

\begin{itemize}
\item (a) $z$ есть наибольший общий делитель $x$ и $y$;

\item (b) $z$ есть остаток от деления $x$ на $y$, и $y$ отлично от нуля;

\item (c) $x$ есть степень некоторого простого числа.
\end{itemize}

\begin{Solution}
Обозначим $a \vdots b \leftrightharpoons \exists k \left[kb = a \land b \not= 0\right]$. Тогда утверждение (a) выглядит так:
\[
	\forall z' \left[x \vdots z' \land y \vdots z' \supset z' \le z\right]
\]

Утверждение (b) выглядит так:
\[
	\exists k \left[(x = ky + z) \land (y \not= 0)\right]
\]

Обозначим:
\[
	Prime\left(x\right) \leftrightharpoons \forall k\left[\left(k = 0\right)\lor\left(x\not= S0\right)\land\left(x\not= 0\right)\land\left(x\vdots k \supset \left(k=S0\right)\lor\left(k=x\right)\right)\right]
\]

Тогда последнее утверждение можно записать так:
\[
	\left(x=0\right)\lor \forall k'\forall k'' \left[Prime\left(k'\right)\land Prime\left(k''\right)\land\left(x\vdots k'\right)\land\left(x\vdots k''\right)\supset\left(k'=k''\right)\right]
\]
\end{Solution}

\paragraph{Задание 16.} Будем рассматривать язык первого порядка, сигнатура, которого содержит только 3 одноместных предикатных символа Т, Пр, Пл и 2 два двуместных предикатных символа $\in, =$. Зададим интерпритацию этого языка: носителем является множество всех точек, прямых и плоскостей 3-мерного евклидова пространства; предикатным символам сопоставлены одноименные предикаты, определенные таким образом:
\begin{itemize}
\item T($x$) : <<$x$ - точка>>,

\item Пр($x$) : <<$x$ - прямая>>,

\item Пл($x$) : <<$x$ - плоскость>>,

\item $x \in y$ : <<$x$ принадлежит (лежит на) $y$>>,

\item $x = y$ : <<$x$ совпадает с $y$>>.
\end{itemize}

Запишите на этом языке формулу, выражающую в данной интерпритации следующее:
\begin{itemize}
\item ($a$) через любые 3 точки, не лежащие на одной прямой, можно провести единтсвенную плоскость;

\item ($b$) если 2 точки прямой лежат в плоскости, то всякая точка этой прямой лежит в этой плоскости;

\item ($c$) 2 прямые, лежащие в одной и той же плоскости, либо имеют одну общую точку, либо не имеют ни одной;
\end{itemize}

\begin{Solution}
Введем следующее обозначение:
\[
	Online\left(x,y,z\right) \leftrightharpoons \exists l\left[\text{Т}\left(x\right)\land \text{Т}\left(y\right)\land\text{Т}\left(z\right)\land\text{Пр}\left(l\right)\land\left(x \in l\right)\land\left(y \in l\right)\land\left(z \in l\right)\right]
\]

Тогда утверждение $(a)$ будет выглядеть следующим образом:
\[
	\forall x \forall y \forall z \exists ! p \left[\text{Пл}\left(p\right) \land \neg Online\left(x,y,z\right)\supset\left(x \in p\right)\land\left(y \in p\right)\land\left(z \in p\right)\right]
\]

Введем обозначения:
\[
	\begin{split}
		& Online\left(x,y,l\right) \leftrightharpoons \left[\text{Т}\left(x\right)\land\text{Т}\left(y\right)\land\text{Пр}\left(l\right)\land\left(x \in l\right)\land\left(y \in l\right)\right] \\
		& Onplate\left(x,y,p\right) \leftrightharpoons \left[\text{Т}\left(x\right)\land\text{Т}\left(y\right)\land\text{Пл}\left(p\right)\land\left(x \in p\right)\land\left(x \in p\right)\right]
	\end{split}
\]

Тогда утверждение $(b)$ выглядит следующим образом:
\[
	\forall l \forall p \left[\exists x \exists y \left[Online\left(x,y,l\right)\land Onplate\left(x,y,p\right)\right]\supset\left(l \in p\right)\right]
\]

Введем обозначение
\[
	\begin{split}
		& Cross\left(x,y\right) \leftrightharpoons \text{Пр}\left(x\right)\land\text{Пр}\left(y\right)\land\exists !p \left[\text{Т}\left(p\right)\land\left(p \in x\right)\land\left(p \in y\right)\right] \\
		& Nocross\left(x,y\right) \leftrightharpoons \text{Пр}\left(x\right)\land\text{Пр}\left(y\right)\land\neg\exists p\left[\text{Т}\left(p\right)\land\left(p \in x\right)\land\left(p \in y\right)\right]
	\end{split}
\]
Утверждение $(c)$ записывается следующим образом:
\[
	\forall l_1 \forall l_2 \left[\left(l_1 \not= l_2\right)\land\exists p \left[\text{Пл}\left(p\right)\land\left(l_1 \in p\right)\land\left(l_2 \in p\right)\right]\supset Cross\left(l_1,l_2\right)\text{ xor }Nocross\left(l_1,l_2\right)\right]
\]
\end{Solution}

\paragraph{Задание 17.} Докажите, что формула истинна во всякой интерпритации при любой оценке, или укажите интерпритацию и оценку, при которой эта формула ложна.
\begin{itemize}
\item $(a)$ $\exists x\forall y P\left(x,y\right) \supset \forall y \exists x P\left(x,y\right)$

\item $(b)$ $\forall y \exists x P\left(x,y\right) \supset \exists x\forall y P\left(x,y\right)$
\end{itemize}

\begin{Solution}
Докажем для пункта $(a)$:
\[
	\begin{split}
		& \exists x \forall y P\left(x,y\right) \supset \forall y \exists x P\left(x,y\right) \sim \forall y \exists x \left(\exists x \forall y P\left(x,y\right) \supset P\left(x,y\right)\right) \sim \\
		& \sim \forall y \exists x \forall x \left(\forall y P\left(x,y\right) \supset P\left(x,y\right)\right) \sim \forall y \exists x \forall x \exists y \left(P\left(x,y\right) \supset P\left(x,y\right)\right) \sim \\
		& \sim \forall y \exists x \forall x \exists y \mathbb{T} \sim \mathbb{T}
	\end{split}
\]

докажем для пункта $(b)$:
\[
	\begin{split}
		& \forall y \exists x P\left(x,y\right) \supset \exists x \forall y P\left(x,y\right) \sim \exists x \forall y \left(\forall y \exists x P\left(x,y\right) \supset P\left(x,y\right)\right) \sim \\
		& \sim \exists x \forall y \exists y \forall x \left(P\left(x,y\right) \supset P\left(x,y\right)\right) \sim \exists x \forall y \exists y \forall x \mathbb{T} \sim \mathbb{T}
	\end{split}
\]
\end{Solution}

\paragraph{Задание 18.} Укажите и обоснуйте (разумно слабые) условия на формулу $A$, переменную $x$ и терм $t$, при которых формула
\[
	\forall x A \supset \left[A\right]_{t}^{x}
\]
истинна во всякой интерпритации при лбой оценке. Если это возможно, укажите конкретные формулу $A$, переменную $x$, терм $t$, интерпритацию и оценку, при которых вормула ложна.

\begin{Solution}
Все выражение ложно в том и только том случае, если посылка истинна, а следствие ложно. Далее так как $t$ - терм, то он является либо предметной переменной, либо $n$-местным функционльным символом, которому при любой интерпритации $M$ и оценке $\nu$ сопоставляется элемент из носителя (как и предметной переменной $x$), но из семантики формулы $\forall x A$ следует, что для любого элемента $\alpha$ из носителя $\left|A\right|_{M,\nu_{\alpha}^{t}} = 1$, следовательно, при любой замене $x$ на $t$ получаем истинностное значение $1$ в интерпритации.
\end{Solution}

\end{document}
