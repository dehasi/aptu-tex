\documentclass[a4paper,12pt]{article}

\usepackage[T2A]{fontenc} 
\usepackage[utf8]{inputenc}
\usepackage[english,russian]{babel}
\usepackage{listings}
\usepackage[dvips]{graphicx}
\usepackage{indentfirst}
\usepackage{color}
\usepackage{hyperref}
\usepackage{amsmath}
\usepackage{amssymb}
\usepackage{geometry}
\geometry{left=1.5cm}
\geometry{right=1.5cm}
\geometry{top=1cm}
\geometry{bottom=2cm}

\graphicspath{{images/}}

\begin{document}
\sloppy

\lstset{
	basicstyle=\small,
	stringstyle=\ttfamily,
	showstringspaces=false,
	columns=fixed,
	breaklines=true,
	numbers=right,
	numberstyle=\tiny
}

\newtheorem{Def}{Определение}[section]
\newtheorem{Th}{Теорема}
\newtheorem{Lem}[Th]{Лемма}
\newenvironment{Proof}
	{\par\noindent{\bf Доказательство.}}
	{\hfill$\scriptstyle\blacksquare$}
\newenvironment{Solution}
	{\par\noindent{\bf Решение.}}
	{\hfill$\scriptstyle\blacksquare$}


\begin{flushright}
	Кринкин М. Ю. группа 504 (SE)
\end{flushright}

\section{Домашнее задание 2}

\paragraph{Задание 2.} Проверить насколько материализация влияет на время выполнения запросов.

Необходимо выполнить запросы с использованием представлений и без них и на основе практических наблюдений за серией запросов ответить на следующие вопросы:
\begin{itemize}
\item Использование каких представлений позволяет улучшить время выполнения каких запросов (составить таблицу)?

\item При какой наполнености базы целесообразно использовать представления в запросах?
\end{itemize}

В качестве представлений предложено выбрать следующие:
\begin{lstlisting}
CREATE VIEW Towns AS
SELECT DISTINCT Town
FROM Players;
\end{lstlisting}

\begin{lstlisting}
CREATE VIEW CPlayers AS
SELECT PlayerNo, LeagueNo
FROM Players
WHERE LeagueNo IS NOT NULL;
\end{lstlisting}

\begin{lstlisting}
CREATE VIEW RESIDENT (Town, Number) AS
SELECT Town, COUNT(*)
FROM Players
GROUP BY Town;
\end{lstlisting}

\begin{lstlisting}
CREATE VIEW SFD_FOLK (PlyNo, Name, Initials, Born) AS
select PlayerNo, Name, Initials, BirthYear
from Players
where Town = 'Stratford';
\end{lstlisting}

В качестве запросов не имеет сымсла выбирать запросы которые никак не связаны с представлениями, поэтому из приведенных представлений будем пользоваться только представлением SFD\_FOLK (причем реализованное имеено в виде таблицы) и следующие запросы:

\begin{lstlisting}
SELECT PlayerNo, Street + ' ' + HouseNo AS Address
FROM Players
WHERE Town = 'Stratford';
\end{lstlisting}

Этот запрос с использованием представления:

\begin{lstlisting}
SELECT PlyNo, Street + ' ' + HouseNo AS Address
FROM SFD_FOLK, Players
WHERE PlyNo = PlayerNo;
\end{lstlisting}

и ради эксперемента перепишем его еще так:

\begin{lstlisting}
SELECT PlayerNo, Street + ' ' + HouseNo AS Address
FROM Players, SFD_FOLK
WHERE PlyNo = PlayerNo;
\end{lstlisting}

Рассмотрим сводную таблицу для этих запросов:

\begin{tabular}[t]{|c|c|c|c|c|}
\hline
Наполнение базы &    1 &    2 &    3 &    4 \\
\hline
Запрос 1        &   45 &   40 &  338 & 1340 \\
\hline
Запрос 2        &   77 &  240 &   88 &  104 \\ 
\hline
Запрос 3        &   77 &  240 &   91 &  100 \\
\hline
\end{tabular}

Видно, что представления дают заметный выигрыш при выполнении запроса.

Следующими рассмотрим следующий запрос без использования представлений:

\begin{lstlisting}
SELECT PlayerNo, LeagueNo
FROM Players
WHERE Town = 'Stratford'
ORDER BY LeagueNo;
\end{lstlisting}

и запросы с использование представлений:

\begin{lstlisting}
SELECT PlyNo, LeagueNo
FROM SFD_FOLK, Players
WHERE PlyNo = PlayerNo
ORDER BY LeagueNo;
\end{lstlisting}

\begin{lstlisting}
SELECT PlayerNo, LeagueNo
FROM Players, SFD_FOLK
WHERE PlyNo = PlyerNo
ORDER BY LeagueNo;
\end{lstlisting}

Сводная таблица результатов:

\begin{tabular}[t]{|c|c|c|c|c|}
\hline
Наполнение базы  &    1 &    2 &    3 &    4 \\
\hline
Запрос 1         &   55 &   54 &  340 & 1384 \\
\hline
Запрос 2         &  135 &  290 &  130 &  136 \\
\hline
Запрос 3         &  140 &  290 &  127 &  139 \\
\hline
\end{tabular}

И опять представления дают большой выигрыш, и уже не трудно проследить, что использование представления помогает при больших объемах данных.

Теперь рассмотрим ледующие запросы, запрос без представлений:

\begin{lstlisting}
SELECT PlayerNo, Town, BirthYear
FROM Players
WHERE (Town = 'Stratford' OR BirthYear = 1963)
AND NOT (Town = 'Stratford' AND BirthYear = 1963);
\end{lstlisting}

и запросы с использованием представлений (в данном случае запросов два потому, что одним простым запросом этого сделать не получилось, потому что в учебной базе данных ограниченный синтаксис SQL):

\begin{lstlisting}
SELECT PlyNo, 'Stratford', Born
FROM SFD_FOLK
WHERE Born <> 1963;
\end{lstlisting}

\begin{lstlisting}
SELECT PlayerNo, Town, BirthYear
FROM Players
WHERE BirthYear = 1963 AND Town <> 'Stratford';
\end{lstlisting}

Сводная таблица результатов:

\begin{tabular}[t]{|c|c|c|c|c|}
\hline
Наполнение базы &    1 &    2 &    3 &    4 \\
\hline
Запрос 1        &   45 &   55 &  337 & 1394 \\
\hline
Запрос 2        &   40 &   35 &   34 &   43 \\
\hline
Запрос 3        &   32 &   34 &  330 & 1360 \\
\hline
\end{tabular}

Тут суммарное время выполнения запросов 2 и 3 сравнимо со временем выполнения запроса 1, но тем не менее выигрыш от использования представлений также значителен.

\paragraph{Вывод:} В данной работе мы попробовали выполнять запросы с использованием представлений и без них, при этом заметен выигрыш
по скорости выполнения при использовании представлений. По итогам работы можно сделать следующие выводы:

\begin{itemize}
\item Представления позволяют сократить время выполнения запросов, при использовании его в пересечении (см таблицы 1 и 2),
за счет раннего скоращения объемов данных, т. е. если часть необходимого фильтра реализованна в представлении, то при использовании
пересечения с представлением заранее сократится объем читаемых и хранимых в памяти данных.

\item В качестве представления удобно оформлять запросы, к которым часто происходит обращение (см таблицу 3, запрос 2, предобработка
ускоряет выполнение запроса, поэтому обращение к представлению быстрее, чем исполнение запроса).

\item Наконец, при оформлении запросов в виде View, сложные запросы с большим количеством вложенных select-ов могут быть переоформлены гораздо проще и красивее.
\end{itemize}
\end{document}
