\documentclass[a4paper,12pt]{article}

\usepackage[T2A]{fontenc} 
\usepackage[utf8]{inputenc}
\usepackage[english,russian]{babel}
\usepackage{listings}
\usepackage[dvips]{graphicx}
\usepackage{indentfirst}
\usepackage{color}
\usepackage{hyperref}
\usepackage{amsmath}
\usepackage{amssymb}
\usepackage{geometry}
\geometry{left=1.5cm}
\geometry{right=1.5cm}
\geometry{top=1cm}
\geometry{bottom=2cm}

\graphicspath{{images/}}

\begin{document}
\sloppy

\lstset{
	basicstyle=\small,
	stringstyle=\ttfamily,
	showstringspaces=false,
	columns=fixed,
	breaklines=true,
	numbers=right,
	numberstyle=\tiny
}

\newtheorem{Def}{Определение}[section]
\newtheorem{Th}{Теорема}
\newtheorem{Lem}[Th]{Лемма}
\newenvironment{Proof}
	{\par\noindent{\bf Доказательство.}}
	{\hfill$\scriptstyle\blacksquare$}
\newenvironment{Solution}
	{\par\noindent{\bf Решение.}}
	{\hfill$\scriptstyle\blacksquare$}


\begin{flushright}
	Кринкин М. Ю. группа 504 (SE)
\end{flushright}

\section{Домашнее задание 2}

\paragraph{Задание 2.} Проверить насколько материализация влияет на время выполнения запросов.

Необходимо выполнить запросы с использованием представлений и без них и на основе практических наблюдений за серией запросов ответить на следующие вопросы:
\begin{itemize}
\item Использование каких представлений позволяет улучшить время выполнения каких запросов (составить таблицу)?

\item При какой наполнености базы целесообразно использовать представления в запросах?
\end{itemize}

В качестве представлений предложено выбрать следующие:
\begin{lstlisting}
CREATE VIEW Towns AS
SELECT DISTINCT Town
FROM Players;
\end{lstlisting}

\begin{lstlisting}
CREATE VIEW CPlayers AS
SELECT PlayerNo, LeagueNo
FROM Players
WHERE LeagueNo IS NOT NULL;
\end{lstlisting}

\begin{lstlisting}
CREATE VIEW RESIDENT (Town, Number) AS
SELECT Town, COUNT(*)
FROM Players
GROUP BY Town;
\end{lstlisting}

В качестве запросов не имеет сымсла выбирать запросы которые никак не связаны с представлениями, поэтому из приведенных представлений будем пользоваться только представлением SFD\_FOLK и следующие запросы:

\begin{lstlisting}
SELECT PlayerNo, Street + ' ' + HouseNo AS Address
FROM Players
WHERE Town = 'Stratford';
\end{lstlisting}

Этот запрос с использованием представления:

\begin{lstlisting}
SELECT PlyNo, Street + ' ' + HouseNo AS Address
FROM SFD_FOLK, Players
WHERE PlyNo = PlayerNo;
\end{lstlisting}

и ради эксперемента перепишем его еще так:

\begin{lstlisting}
SELECT PlayerNo, Street + ' ' + HouseNo AS Address
FROM Players, SFD_FOLK
WHERE PlyNo = PlayerNo;
\end{lstlisting}

Рассмотрим сводную таблицу для этих запросов:

\begin{tabular}[t]{|c|c|c|c|c|}
\hline
Наполнение базы &    1 &    2 &    3 &    4 \\
\hline
Запрос 1        &  172 &   64 &  338 & 1263 \\
\hline
Запрос 2        &   77 &  132 &  396 & 1307 \\ 
\hline
Запрос 3        &  118 &  118 &  387 & 1275 \\
\hline
\end{tabular}

В данном случае влияние представлений не очень помогает, в конце работы я немного перепишу представление, чтобы оно более соответствовало запросу и не требовалось привлекать JOIN с таблицей Players.

Следующими рассмотрим следующий запрос без использования представлений:

\begin{lstlisting}
SELECT PlayerNo, LeagueNo
FROM Players
WHERE Town = 'Stratford'
ORDER BY LeagueNo;
\end{lstlisting}

и запросы с использование мпредставлений:

\begin{lstlisting}
SELECT PlyNo, LeagueNo
FROM SFD_FOLK, Players
WHERE PlyNo = PlayerNo
ORDER BY LeagueNo;
\end{lstlisting}

\begin{lstlisting}
SELECT PlayerNo, LeagueNo
FROM Players, SFD_FOLK
WHERE PlyNo = PlyerNo
ORDER BY LeagueNo;
\end{lstlisting}

Сводная таблица результатов:

\begin{tabular}[t]{|c|c|c|c|c|}
\hline
Наполнение базы  &    1 &    2 &    3 &    4 \\
\hline
Запрос 1         &   63 &   54 &  356 & 1277 \\
\hline
Запрос 2         &  148 &  163 &  463 & 1562 \\
\hline
Запрос 3         &  129 &  126 &  434 & 1382 \\
\hline
\end{tabular}

Тут представления даже не помогли, а наоборот испортили, и кстати, тут можно увидеть еще одно интересное явление, хотя 2 и 3 запрос очень похожи, время их работы заметно различается, вероятно это связано с тем, какая из таблиц Plaers или SFD\_FOLK просматривается во внешнем цикле, а какая во внутреннем.

Теперь рассмотрим ледующие запросы, запрос без представлений:

\begin{lstlisting}
SELECT PlayerNo, Town, BirthYear
FROM Players
WHERE (Town = 'Stratford' OR BirthYear = 1963)
AND NOT (Town = 'Stratford' AND BirthYear = 1963);
\end{lstlisting}

и запросы с использованием представлений (в данном случае запросов два потому, что одним простым запросом этого сделать не получилось, потому что в учебной базе данных ограниченный синтаксис SQL):

\begin{lstlisting}
SELECT PlyNo, 'Stratford', Born
FROM SFD_FOLK
WHERE Born <> 1963;
\end{lstlisting}

\begin{lstlisting}
SELECT PlayerNo, Town, BirthYear
FROM Players
WHERE BirthYear = 1963 AND Town <> 'Stratford';
\end{lstlisting}

Сводная таблица результатов:

\begin{tabular}[t]{|c|c|c|c|c|}
\hline
Наполнение базы & & & & \\
\hline
Запрос 1        &   62 &   62 &  343 & 1265 \\
\hline
Запрос 2        &   56 &   54 &  345 & 1290 \\
\hline
Запрос 3        &   32 &   32 &  363 & 1299 \\
\hline
\end{tabular}

Тут поражение представлений совсем полное, вероятно, в данных простых запросах представления не ускоряют исполнения, но кроме того, как видно из предыдущего задания база данных начиная с некоторого заполнения использует индексы, поэтому сами запросы выполняются достаточно быстро.

Рассмотрим теперь другое представление:

\begin{lstlisting}
CREATE VIEW FSD_FOLK2(PlyNo, Stt, HsNo) AS
SELECT PlayerNo, Street, HouseNo
FROM Players
WHERE Town = 'Stratford';
\end{lstlisting}

Тогда самый первый рассмотренный запрос преобразуется следующим образом:

\begin{lstlisting}
SELECT PlyNo, Stt + ' ' + HsNo
FROM SFD_FOLK2;
\end{lstlisting}

Исполнение запроса на наиболшем наполнении базы занимает 1277 мс, что тоже не увеличивает производительность.

\paragraph{Вывод:} В данном случае представления скорее ухудшили производительность, но тем не менее представления действиетльно полезны по целому ряду причин:

\begin{itemize}
\item С помощью представлений можно упростить запросы к базе данных (более красивый и простой код)

\item Если нет возможности использовать индексы, то удачным решением будет организовать частые запросы и подзапросы в представления (например, на практике работя с учебной базой данных World часто приходилось считать городское население страны в различных запросах, поэтому хорошо было сделать представление)
\end{itemize}
\end{document}
