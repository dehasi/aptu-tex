\chapter{Снова о ящиках}

По материалам лекции от 3 ноября 2011 года.

\subsection{Производящие функции и ящики}

Рассмотрим задачу о распределении неразличимых предметов по различимым ящикам. Для этого рассмотрим еще раз, что происходит с обыкновенными производящими функциями при перемножении:

\[
	\begin{split}
		& f\left(x\right) = a_0 + a_1 x + a_2 x^2 + ... \\
		& g\left(x\right) = b_0 + b_1 x + b_2 x^2 + ... \\
		& h\left(x\right) = f\left(x\right) g\left(x\right) = c_0 + c_1 x + c_2 x^2 \\
		& c_n = \sum_{i=0}^n a_i b_{n-i}
	\end{split}
\]

Пусть теперь у нас есть $n$ предметов и мы желаем разложить их по двум ящикам, тогда, если педметы не различимы, нам важно лишь количество предметов в каждом ящике, т. е. получаются варианты: один ящик пустой, все $n$ предметов попали во второй, в первом ящике 1 предмет, во втором $n-1$, в первом ящике 2 предмета, во втором $n-2$ и тд. Всего количество вариантов в точности $n+1$, и каждый вариант получается только одним способом, т. е. в точности произведение обыкновенных производящих функций при условии $a_i = b_i = 1$.

Экстраполяция на большее число ящиков получается за счет перемножения большего числа производящих функций, теперь сразу к технике, производящая функция для следующей последовательности:
\[
	1, 1, 1, 1, 1, 1, 1, 1, 1, 1, 1, 1 ...
\]
(* когда натренируешься, так прикольно рисовать эти единички=)).

будет следующей:
\[
	f\left(x\right) = \frac{1}{1-x}
\]

таким образом для задачи разложения неразличимых предметов по $k$ различимым ящикам без ограничений на количество предметов в ящике получаем:
\[
	r\left(x\right) = \left(f\left(x\right)\right)^k = \frac{1}{\left(1-x\right)^k} = \sum_{n=0}^{\infty} \binom{n+k-1}{n} x^n
\]
т. е. получили уже известное нам решение - сочетания с повторениями.

Теперь добавим в задачу ограничения, пусть в каждом ящике должен лежать хотябы 1 предмет, тогда рассматриваем последовательность:
\[
	0, 1, 1, 1, 1, 1, 1, ...
\]
производящая функция для нее:
\[
	f\left(x\right) = \frac{x}{1-x}
\]
и решение тогда приобретет вид:
\[
	r\left(x\right) = \left(f\left(x\right)\right)^k = x^k \frac{1}{\left(1-x\right)^k} = x^k \sum_{n=0}^{\infty} \binom{n+k-1}{n} x^n = \sum_{n=0}^{\infty} \binom{n+k-1}{n} x^{n+k}
\]
решение опять же нам известное и раньше.

Теперь перейдем к более сложной задаче, решение которой также было получено ранее, но в очень ограниченной форме. Пусть в каждом ящике не может лежать больше $m$ предметов, тогда рассматриваем следующие последовательности:
\[
	\underbrace{1, 1, 1, ..., 1,}_{m} 0, 0, 0, ...
\]
Производящая функция для такой последовательности есть просто (ну не совсем просто) многочлен:
\[
	f\left(x\right) = 1 + x + x^2 + x^3 + ... + x^m
\]

а результирующая функция:
\[
	r\left(x\right) = \left(f\left(x\right)\right)^k = \left(1 + x + x^2 + x^3 + ... + x^m\right)^k
\]
Получаем уже чисто вычислительную задачу, если число $m$ не очень большое то иногда даже можно руками посчитать, например, пусть в каждом ящике должно быть не более 1 предмета, тогда имеем:
\[
	f\left(x\right) = 1 + x
\]
и результат:
\[
	r\left(x\right) = \left(f\left(x\right)\right)^k = \left(1+x\right)^k = \sum_{n=0}^{k} \binom{k}{n} x^n
\]
получили опять известный нам результат.

\subsection{Композиция производящих функций}

Иногда после того, как мы получили разбиение линейно-упорядоченного множества на блоки (см предыдщуий пункт), над блоками необходимо совершить комбинаторное действие, в этом случае интуитивно поянтно, что должна использоваться композиция производящих функций. Но прежде чем определять такую композицию рассмотрим задачу о марках:

Пусть имеются марки достоинством $4, 6, 10$ требуется наклеить марок на $n$ единиц (порядок наклейки важен, нужно ровно $n$ единиц). Напишем рекуррентное соотношение:
\[
	F\left(n\right) = F\left(n-4\right) + F\left(n-6\right) + F\left(n-10\right)
\]
при начальных условиях:
\[
	\begin{split}
		& F\left(0\right) = 1 \\
		& F\left(n\right) = 0, n < 0
	\end{split}
\]

Теперь воспользуемся известной техникой, построим производящую функцию для решения данной задачи, исходя из того, что последовательность:
\[
	a_0, a_1, a_2, a_3, ...
\]
является решением получаем
\[
	\begin{split}
		& \sum a_{n+10} x^{n+10} = x^4 \sum a_{n+6} x^{n+6} + x^6 \sum a_{n+4} x^{n+4} + x^{10} \sum a_n x^{n} \Leftrightarrow \\
		& \Leftrightarrow f\left(x\right) - 1 - x^4 - x^6 - x^8 = x^4 \left[f\left(x\right) - 1 - x^4\right] + x^6 \left[f\left(x\right) - 1\right] + x^{10} f\left(x\right) \Leftrightarrow \\
		& \Leftrightarrow f\left(x\right) \left[1 - x^4 - x^6 - x^{10}\right] = 1 \Leftrightarrow f\left(x\right) = \frac{1}{1 - \left(x^4 + x^6 + x^{10}\right)}
	\end{split}
\]
не правда ли чудесное совпадение?!?!?!? Но давай всмотримся в полученную функцию, замечаем, что если в производящей функции $\frac{1}{1 - x}$ заменить $x$ на $x^4 + x^6 + x^{10}$ получаем наше выражение, чтобы это значило?

\begin{Def}
Пусть имеются две производящие функции:
\[
	\begin{split}
		& f\left(x\right) = 1 + a_1 x + a_2 x^2 + a_3 x^3 + ... \\
		& g\left(x\right) = 1 + b_1 x + b_2 x^2 + b_3 x^3 + ... \\
	\end{split}
\]
тогда композицией производящих функций называется:
\[
	s\left(x\right) = f\left(g\left(x\right)\right) = 1 + a_1 f\left(x\right) + a_2 \left(f\left(x\right)\right)^2 + a_3 \left(f\left(x\right)\right)^3 + ...
\]
\end{Def}

Теперь посмотрев на определение становится понятно, что в нашем примере в результирующей функции возле $n$-ого коэффициента будет стоять $\left(x^4 + x^6 + x^10\right)^n$ т. е. в точности число способов получить из марок достоинством $4, 6, 10$ сумму в $n$, собственно то, что мы и получили ранее.

Таким образом суть композиции обыкновенных производящих функций в следующем, сначала мы определенным образом разбиваем множество, потом над полученными блоками совершаем некоторое действие. И вот число способов так наворотить как раз и определяется композицией производящих функций.
