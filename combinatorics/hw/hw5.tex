\documentclass[a4paper,12pt]{article}

\usepackage[T2A]{fontenc} 
\usepackage[utf8]{inputenc}
\usepackage[english,russian]{babel}
\usepackage{listings}
\usepackage[dvips]{graphicx}
\usepackage{indentfirst}
\usepackage{color}
\usepackage{hyperref}
\usepackage{amsmath}
\usepackage{amssymb}
\usepackage{geometry}
\geometry{left=1.5cm}
\geometry{right=1.5cm}
\geometry{top=1cm}
\geometry{bottom=2cm}

\graphicspath{{images/}}

\begin{document}
\sloppy

\lstset{
	basicstyle=\small,
	stringstyle=\ttfamily,
	showstringspaces=false,
	columns=fixed,
	breaklines=true,
	numbers=right,
	numberstyle=\tiny
}

\newtheorem{Def}{Определение}[section]
\newtheorem{Th}{Теорема}
\newtheorem{Lem}[Th]{Лемма}
\newenvironment{Proof}
	{\par\noindent{\bf Доказательство.}}
	{\hfill$\scriptstyle\blacksquare$}
\newenvironment{Solution}
	{\par\noindent{\bf Решение.}}
	{\hfill$\scriptstyle\blacksquare$}


\begin{flushright}
	Кринкин М. Ю. группа 504 (SE)
\end{flushright}

\section{Домашнее задание 5}

\paragraph{Задание 1.} Мы положили тысячу рублей в банк под пять процентов годовых. В начале каждого года мы докладываем пятьсот рублей на счет. Сколько денег будет на счете через $n$ лет?

\begin{Solution}
Размер счета описывается простым реккурентным соотношением $a_n = 1.05 a_{n-1} + 500$, при начальных условиях $a_0 = 1000$. Посмотрим на первые несколько членов последовательности:
\begin{itemize}
\item $a_0 = 1000$

\item $a_1 = 1000 \cdot 1.05 + 500$

\item $a_2 = a_1 \cdot 1.05 + 500 = \left(1000\cdot 1.05 + 500\right) \cdot 1.05 + 500 = 1000 \cdot 1.05^2 + 500 \cdot \left(1.05 + 1\right)$

\item $a_i = a_0 \cdot 1.05^i + \sum_{k=0}^{i-1} 500 \cdot 1.05^k= a_0 \cdot 1.05^i + 500 \frac{1-1.05^{i}}{1-1.05}$

\end{itemize}
\end{Solution}

\paragraph{Задание 2.} Пусть $2n$ игроков принимают участие в тенисном турнире. Определить с использованием рекуррентного соотношения число $a_n$ различных пар, которые можно сформировать для $n$ матчей первого круга.

\begin{Solution}
\end{Solution}

\paragraph{Задание 3.} Определить с помощью рекуррентного соотношения число $a_n$ областей, на которые забивается плоскость с помощью $n$ окружностей так, что любые из этих окружностей пересекаются ровно по двум точкам и что никакие из трех окружностей не имеют общей точки пересечения.

\begin{Solution}
Для начал рассмотрим первые несколько таких разбиений, например до 5 окружностей. Получаем следующую последовательность $a_1 = 2$, $a_2 = 4$, $a_3 = 8$, $a_4 = 14, 22$, и рассмотрим разницу между соседними числами последовательности, получаем следующую последовательность $2, 4, 6, 8$. Идея рассмотрения разницы заключается в следующем, каждая вновь добавленая окружность уеличивает количество областей ровно на столько, сколько старых областей она пересекает (т. е. каждая пересеченная область делится линией окружности попалам), каждая следующая окружность как видно пересекает на 2 области больше чем предыдущая, получаем рекуррентное соотношение:
\[
	a_n = a_{n-1} + 2 \cdot (n-1) = a_{n-1} + (a_{n-1} - a_{n-2}) + 2 = 2 \cdot a_{n-1} - a_{n-2} + 2
\]
\end{Solution}

\paragraph{Задание 4.} Рассмотрим плоскость $\left(x,y\right)$. Предположим, что мы можем ходить по плоскости, делаяя шаг верх $U$, шаг право $R$ и шаг влево $L$ на единицу длины так, чтобы шаг $R$ не следовал за шагом $L$ и наоборот (так называемые решеточные пути на плоскости). Пусть $a_n$, $a_0=1$ - число таких путей после $n$ шагов. Найти $a_n$.

\begin{Solution}
Как я понял задание, $a_n$ - количество всех путей длинны $n$, исходящих из одной точки и удовлетворяющих условиям выше. Теперь введем несколько обозначений, $U_n$ - количество путей длинны $n$ из данной точки, при условии, что пришли мы в нее снизу (т. е. из данной точки мы можем пойти в три стороны), а $L_n,R_n$ - количество путей из данной точки при условии, что мы пришли в нее слева или справа, соответственно. Тогда условие задачи состоит в отыскании $U_n$. Составим рекуррентные соотношения:
\[
	\begin{split}
		& U_n = R_{n-1} + L_{n-1} + U_{n-1} \\
		& R_n = R_{n-1} + U_{n-1} \\
		& L_n = L_{n-1} + U_{n-1}
	\end{split}
\]
Из соотношений выше видно, что $R_n$ и $L_n$ - симметричны, и соответственно $R_n = L_n$, т. е. соотношения преобразуются следующим образом:
\[
	\begin{split}
		& U_n = 2 R_{n-1} U_{n-1} \\
		& R_n = R_{n-1} + U_{n-1}
	\end{split}
\]
Теперь выразим $R_{n-1}$ из первого выражения и подставим во второе:
\[
	\begin{split}
		& R_{n-1} = \frac{U_n - U_{n-1}}{2} \Rightarrow \frac{U_{n+1} - U_n}{2} = \frac{U_n - U_{n-1}}{2} + U_{n-1} \Leftrightarrow \\
		& \Leftrightarrow U_{n+1} = 2 U_n + U_{n-1}
	\end{split}
\]
Видно, что требуется уже 2 начальных условия, одно $U_0 = 1$, а второе не трудно построить $U_1 = 3$.
Составляем характеристическое уравнение:
\[
	\begin{split}
		& r^2 - 2 r - 1 = 0 \\
		& r_{1,2} = 1 \pm \sqrt{2}
	\end{split}
\]
Решение ищем в виде $U_n = C_1 r_1^n + C_2 r_2^n = C_1 {\left(1+\sqrt{2}\right)}^n + C_2 {\left(1 - \sqrt{2}\right)}^n$, исходя из начальных условий получаем систему уравнений:
\[
	\begin{split}
		& 1 = C_1 + C_2 \\
		& 3 = C_1 \left(1 + \sqrt{2}\right) + C_2 \left(1 - \sqrt{2}\right)
	\end{split}
\]
откуда получаем:
\[
	\begin{split}
		& C_1 = \frac{\sqrt{2} + 2}{2\sqrt{2}} \\
		& C_2 = \frac{\sqrt{2} - 2}{2 \sqrt{2}}
	\end{split}
\]
\end{Solution}

\paragraph{Задание 5.} Космический зонд обнаружил, что органическое вещество на Марсе имеет ДНК, состоящее из пяти символов $\left(a,b,c,d,e\right)$; четыре пары симолов - $ce, cd, ed, ee$ - никогда не встречаются в марсианских ДНК, однако любая цепочка, не содержащая этих пар возможна. Порядок букв в цепочке важен, поэтому, например, цепочка $bbdca$ возможна, а $bbcda$ - нет. Найти рекуррентные соотношения, которым удовлетворяют эти цепочки слов.

\begin{Solution}
Ввведем обозначения, количество цепочек ДНК длинны $n$ которые могут начинаться слюбого символа обозначем через $I_n$, цепочки ДНК длинны $n$ которые не могут начинаться с символов $e$ или $d$ обозначим через $I_n^{e,d}$. Тогда можно сказать, что количество всех цепочек ДНК длинны $n$ складывается из количества цепочек ДНК длинны $n-1$, перед которыми стоят символы $a$, $b$ или $d$, плюс количество цепочек ДНК длинны $n-1$ перед которыми стоят символы $c$ или $e$, т. е. те цепочки, которые не могут начинаться с $e$ или $d$, в итоге имеем:
\[
	\begin{split}
		& I_n = 3 I_{n-1} + 2 I_{n-1}^{e,d} \\
		& I_{n}^{e,d} = 2 I_{n-1} + I_{n-1}^{e,d}
	\end{split}
\]
Второе уравнение вытекает из того, что цепочки, которые не могут начинаться с $e$ или $d$, могут начинаться с $a$, $b$ или $c$, причем на цепочки начинающиеся с $c$ опять же накладываются ограничения.
Теперь выразим из первого уравнения $I_{n-1}^{e,d}$:
\[
	I_{n-1}^{e,d} = \frac{I_{n-1} - 3 I_n}{2}
\]
Подставляем во второе уравнение и в конечном итоге получаем:
\[
	I_{n+1} = I_{n-1} + 4 I_n
\]
Далее задаем начальные условия:
\[
	\begin{split}
		& I_0 = 1 \\
		& I_1 = 5 \\
	\end{split}
\]
далее проверяем правильна ли формула для следующих значений:
\[
	\begin{split}
		& I_2 = 1 + 4 \cdot 5 = 21 \\
		& I_3 = 5 + 4 \cdot 21 = 89
	\end{split}
\]
(Оба варианта я проверил, действительно, все правильно).
Характеристическое уравнение:
\[
	\begin{split}
		& r^2 - 4 r - 1 = 0 \\
		& r_{1,2} = 2 \pm \sqrt{5}
	\end{split}
\]
Ищем общее решение в виде $I_n = C_1 r_1^n + C_2 r_2^n = C_1 {\left(2 + \sqrt{5}\right)}^n + C_2 {\left(2 - \sqrt{5}\right)}^n$, для этого решем систему:
\[
	\begin{split}
		& 1 = C_1 + C_2 \\
		& 5 = C_1 \left(2 + \sqrt{5}\right) + C_2 \left(2 - \sqrt{5}\right)
	\end{split}
\]
получаем решение:
\[
	\begin{split}
		& C_1 = \frac{\sqrt{5} + 3}{2\sqrt{5}} \\
		& C_2 = \frac{\sqrt{5} - 3}{2\sqrt{5}}
	\end{split}
\]
\end{Solution}

\end{document}
