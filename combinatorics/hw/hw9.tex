\documentclass[a4paper,12pt]{article}

\usepackage[T2A]{fontenc} 
\usepackage[utf8]{inputenc}
\usepackage[english,russian]{babel}
\usepackage{listings}
\usepackage[dvips]{graphicx}
\usepackage{indentfirst}
\usepackage{color}
\usepackage{hyperref}
\usepackage{amsmath}
\usepackage{amssymb}
\usepackage{geometry}
\geometry{left=1.5cm}
\geometry{right=1.5cm}
\geometry{top=1cm}
\geometry{bottom=2cm}

\graphicspath{{images/}}

\begin{document}
\sloppy

\lstset{
	basicstyle=\small,
	stringstyle=\ttfamily,
	showstringspaces=false,
	columns=fixed,
	breaklines=true,
	numbers=right,
	numberstyle=\tiny
}

\newtheorem{Def}{Определение}[section]
\newtheorem{Th}{Теорема}
\newtheorem{Lem}[Th]{Лемма}
\newenvironment{Proof}
	{\par\noindent{\bf Доказательство.}}
	{\hfill$\scriptstyle\blacksquare$}
\newenvironment{Solution}
	{\par\noindent{\bf Решение.}}
	{\hfill$\scriptstyle\blacksquare$}


\begin{flushright}
	Кринкин М. Ю. группа 504 (SE)
\end{flushright}

\section{Домашнее задание 9}

\paragraph{Задание 1.} Сколькими способами можно выбрать из урны 5 шаров красного, синего, белого и черного цвета при условии, что красные шары выбираются по одному за раз, синего - по 4 за раз, белого - по 5 за раз и черного - по 3 за раз?

\begin{Solution}
Сопоставим выборам красных шаров производящую функцию:
\[
	f_1\left(x\right) = 1+x+x^2+x^3+x^4+...
\]
выборам синих шаров функцию:
\[
	f_2\left(x\right) = 1+x^4+x^8+x^{12}+...
\]
выборам белых шаров функцию:
\[
	f_3\left(x\right) = 1+x^5+x^{10}+x^{15}+...
\]
выборам черных сопоставим функцию:
\[
	f_4\left(x\right) = 1+x^3+x^6+x^9+..
\]
так как на коэффициент при $x^5$ не повлияют слогаемые со со степенями больше 5, ограничим производящие функции:
\[
	\begin{split}
		& f_1\left(x\right)=1+x+x^2+x^3+x^4+x^5 \\
		& f_2\left(x\right)=1+x^4 \\
		& f_3\left(x\right)=1+x^5 \\
		& f_4\left(x\right)=1+x^3
	\end{split}
\]
произведение приведенных функций:
\[
	\begin{split}
		&res\left(x\right) = 1+x+x^2+2\cdot x^3+3\cdot x^4+4\cdot x^5+3\cdot x^6+4\cdot x^7+5\cdot x^8+5\cdot x^9+4\cdot x^{10}+\\
		&+3\cdot x^{11}+4\cdot x^{12}+3\cdot x^{13}+2\cdot x^{14}+x^{15}+x^{16}+x^{17}
	\end{split}
\]
Таким образом возможны 4 способа:
\[
	\begin{split}
		& \{\{red,red,red,red,red\},\\
		& \{4 \times blue, red\},\\
		& \{5 \times white\},\\
		& \{3 \times black, red, red\}\}
	\end{split}
\]
\end{Solution}

\paragraph{Задание 2.} Сколько существует способов выбора 20 объектов из множества объектов пяти типов при условии, что количество объектов первого типа кратно пяти, второго - трем, объектов третьего типа следует выбирать не более четырех, четвертого - не менее трех, пятого - не более двух?

\begin{Solution}
Аналогично предыдущей задаче, с учетом всех ограничений получаем следующие производящие функции:
\[
	\begin{split}
		&f_1\left(x\right)=1+x^5+x^{10}+x^{15}+x^{20}\\
		&f_2\left(x\right)=1+x^3+x^6+x^9+x^{12}+x^{15}+x^{18}\\
		&f_3\left(x\right)=1+x+x^2+x^3+x^4\\
		&f_4\left(x\right)=\sum_{p=3}^{20} x^p\\
		&f_5\left(x\right)=1+x+x^2
	\end{split}
\]
результат произведения:
\[
	\begin{split}
		&res\left(x\right)=x^3+3\cdot x^4+6\cdot x^5+10\cdot x^6+15\cdot x^7+22\cdot x^8+30\cdot x^9+39\cdot x^{10}+\\
		&+49\cdot x^{11}+60\cdot x^{12}+73\cdot x^{13}+87\cdot x^{14}+102\cdot x^{15}+118\cdot x^{16}+135\cdot x^{17}+\\
		&+154\cdot x^{18}+174\cdot x^{19}+195\cdot x^{20}+...
	\end{split}
\]
Количество способов равно 195.
\end{Solution}

\paragraph{Задание 3.} Раньше в магазинах для взвешивания товаров использовались гири весом 1, 2, 3, 5 и 10 кг. С точки зрения комбинаторики более удачным был бы набор гирь в 1, 2, 4 и 8 кг - он бы, например, позволял единственным образом отвешивать любое число килограмм от 1 до 15. Вообще, имея по одной гире весом в 1, 2, 4, ..., $2^n$ кг, можно единственным образом получить любой вес от 1 до $2^{n-1}$ кг, кладя гири на одну чашу весов. Это утверждение очевидным образом доказывается с использованием записи такого числа в двоичной системе счисления.

\begin{itemize}
\item (a) Доказать тот же факт с использованием производящей функции.

\item (b) Из той же серии: доказать, что с помощью набора гирь в 1, 2, 2, 5, 10, 20, 50, 100, 200, 200, 500 мг можно составить любой вес, равный целому числу мг.
\end{itemize}

\begin{Solution}
Рассмотрим пункт (a). Суть доказать, что:
\[
	\prod_{i=0}^{n}\left(1+x^{2^i}\right)=\sum_{i=0}^{2^{n+1}-1}x^i
\]
покажем это по индукции. База индукции:
\[
	\left(1+x\right)\cdot\left(1+x^2\right)=1+x+x^2+x^3
\]
Индукционное предположение - пусть утверждение справедливо для $n$, покажем, что оно справедливо и для $n+1$:
\[
	\begin{split}
		&\left(1+x^{2^{n+1}}\right)\cdot\sum_{i=0}^{2^{n+1}-1}x^i = \sum_{i=0}^{2^{n+1}-1}x^i + x^{2^{n+1}}\cdot\sum_{i=0}^{2^{n+1}-1} = \\
		&=\sum_{i=0}^{2^{n+1}-1}x^i+\sum_{k=2^{n+1}}^{2\cdot 2^{n+1}-1}x^k = \sum_{i=0}^{2^{n+2}-1}x^i
	\end{split}
\]

Теперь рассмотрим пункт (b). Чисто технически доказательство сводится к произведению следующих производящих функций:
\[
	\begin{split}
		&f_1\left(x\right) = 1+x\\
		&f_2\left(x\right) = 1+x^2+x^4\\
		&f_3\left(x\right) = 1+x^5\\
		&f_4\left(x\right) = 1+x^{10}\\
		&f_5\left(x\right) = 1+x^{20}+x^{40}\\
		&f_6\left(x\right) = 1+x^{50}\\
		&f_7\left(x\right) = 1+x^{100}\\
		&f_8\left(x\right) = 1+x^{200}+x^{400}\\
		&f_9\left(x\right) = 1+x^{500}
	\end{split}
\]
Но учитывая, что максимальная степень в произведении $x^{1110}$, то проверять произведение руками будет довольно трудно, поэтому будем действовать подругому.
Произведение $f_1$, $f_2$ и $f_3$:
\[
	\left(1+x\right)\cdot\left(1+x^2+x^4\right)\cdot\left(1+x^5\right) = 1+x+x^2+x^4+2\cdot x^5+x^6+x^7+x^8+x^9+x^{10}
\]
таким образом произведение первых трех многочленов давет нам возможность получить все веса от 1 до 10. Нетрудно увидеть, что аналогичным образом произведение $f_4$, $f_5$ и $f_6$ позволяет получить все степени кратные 10 от 10 до 100, а так как произведение $f_1$, $f_2$ и $f_3$ позволяет получить любую степень до 10 включительно, то объединив все вместе, сможем получить все степени от 1 до 110. Аналогичным образом произведением функций $f_7$, $f_8$ и $f_9$ сможем получить все степени кратные 100 от 100 до 1000, то объединив с предыдущими результатами сможем получить все степени от 1 до 1110.

\end{Solution}
\end{document}
