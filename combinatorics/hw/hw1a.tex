\documentclass[a4paper,12pt]{article}

\usepackage[T2A]{fontenc} 
\usepackage[utf8]{inputenc}
\usepackage[english,russian]{babel}
\usepackage{listings}
\usepackage[dvips]{graphicx}
\usepackage{indentfirst}
\usepackage{color}
\usepackage{hyperref}
\usepackage{amsmath}
\usepackage{amssymb}
\usepackage{geometry}
\geometry{left=1.5cm}
\geometry{right=1.5cm}
\geometry{top=1cm}
\geometry{bottom=2cm}

\graphicspath{{images/}}

\begin{document}
\sloppy

\lstset{
	basicstyle=\small,
	stringstyle=\ttfamily,
	showstringspaces=false,
	columns=fixed,
	breaklines=true,
	numbers=right,
	numberstyle=\tiny
}

\newtheorem{Def}{Определение}[section]
\newtheorem{Th}{Теорема}
\newtheorem{Lem}[Th]{Лемма}
\newenvironment{Proof}
	{\par\noindent{\bf Доказательство.}}
	{\hfill$\scriptstyle\blacksquare$}
\newenvironment{Solution}
	{\par\noindent{\bf Решение.}}
	{\hfill$\scriptstyle\blacksquare$}


\begin{flushright}
	Кринкин М. Ю. группа 504 (SE)
\end{flushright}

\section{Домашнее задание 1. Исправление}
Доказать следующее реккурентное соотношение для сочетаний с повторениями формально:
\begin{equation}
	\begin{split}
		& \left( \binom{n}{k} \right) = \left( \binom{n-1}{k} \right) + \left( \binom{n}{k-1} \right) \\
		& \left( \binom{n}{1} \right) = \binom{n}{1} = n \\
		& \left( \binom{n}{k} \right) = \binom{1 + k - 1}{k} = 1
	\end{split}
\end{equation}

\begin{Proof}
При доказательстве понадобятся следующие формулы:
\begin{equation}
	\left( \binom{n}{k} \right) = \binom{n+k-1}{k}
	\label{math::relation}
\end{equation}
\begin{equation}
	\binom{n}{k} = \binom{n-1}{k-1} + \binom{n-1}{k}
	\label{math::bin_rec}
\end{equation}
Применим сначала формулу \ref{math::relation}, а затем \ref{math::bin_rec}, после чего произведем свертывание по \ref{math::relation} снова, но в обратном порядке:
\[
	\left( \binom{n}{k} \right) = \binom{n+k-1}{k} = \binom{n+\left(k-1\right)-1}{k-1} + \binom{\left(n-1\right) + \left(k-1\right)}{k} = \left(\binom{n}{k-1}\right) + \left(\binom{n-1}{k}\right)
\]
Остается только проверить граничные условия.
\end{Proof}

\end{document}
