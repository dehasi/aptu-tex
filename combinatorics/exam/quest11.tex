\chapter{Вопрос №11}

Производная и интеграл производящей функции. Построение решений линейных рекуррентных соотношений с переменными коэффициентами с помощью производящих функций.

\section{Производная и интеграл производящей функции}

Пусть у нас имеется опф: $$ f\left(x\right) = a_0 + a_1 x + a_2 x^2 + ... $$
тогда производной этой производящей функции называется производящая функция: $$ f'\left(x\right) = a_1 + 2a_2x + 3a_3x^2 + ... $$
а интегралом этой производящей функции называется производящая функция: $$ \int f\left(x\right) = a_0 x + \frac{a_1}{2} x^2 + \frac{a_2}{3} x^3 + ... $$

Для эпф: $$ F\left(x\right) = a_0 + a_1 x + a_2 \frac{x^2}{2} + a_3 \frac{x^3}{3!} + ... $$ производной будет являться эпф: $$ F'\left(x\right) = a_1 + a_2 x + a_3 \frac{x^2}{2} + a_4 \frac{x^3}{3!} + ... $$, а интегралом соответственное эпф: $$ \int F\left(x\right) = a_0 x + a_1 \frac{x^2}{2} + a_2\frac{x^3}{3!} + ... $$

\section{Рекуррентные соотношения с переменными коэффициентами}

Рассмотрим соотношение $$ \left(n+1\right) a_{n+1} = 2\left(2n+1\right) a_n$$ при начальных условиях $a_0 = 1$. Попробуем найти опф описывающую заданную числовую последовательность, для этого домножим все выражение на $x^n$ и просуммируем:
 $$ \sum_{n=0}^\infty \left(n+1\right) a_{n+1} x^n = 4 \sum_{n=0}^\infty n x^n a_n + 2\sum_{n=0}^\infty a_n $$
откуда по определению производной опф получаем $$ f'\left(x\right) = 4xf'\left(x\right) + 2f\left(x\right) \Leftrightarrow \frac{2}{1-4x} = \frac{f'\left(x\right)}{f\left(x\right)} $$
Решая полученный диффур получаем: $$ f\left(x\right) = \frac{1}{\sqrt{1-4x}}$$
Теперь ищем замкнутую формулу:
\[
	\begin{split}
		& \left(1-4x\right)^{-\frac{1}{2}} = \sum_{n=0}^{\infty} \binom{-1/2}{n}x^n\left(-4\right)^n = \\
		& = 1+ \sum_{n=1}^{\infty}\frac{\left(-1/2\right)\left(-1/2-1\right)...\left(-1/2-n+1\right)}{n!}\left(-4\right)^nx^n = \\
		& = 1+ \sum_{n=1}^{\infty} \frac{1/2 3/2 5/2 ... \left(n-1/2\right)}{n!} 4^n x^n = \\
		& = 1+ \sum_{n=1}^{\infty}\frac{1\cdot3\cdot5\cdot...\cdot\left(2n-1\right)}{n!} 2^n x^n = \\
		& = 1 + \sum_{n=1}^{\infty}\frac{1\cdot3\cdot5\cdot...\cdot\left(2n-1\right)\cdot n! \cdot 2^n}{n!n!}x^n = \\
		& = 1 + \sum_{n=1}^{\infty}\frac{2n!}{n!n!}x^n = \sum_{n=0}^{\infty} \binom{2n}{n}x^n
	\end{split}
\]

\section{Экспоненциальные производящие функции}

Возьмем последовательность $$ a_{n+1} = \left(n+1\right) a_n$$ при начальных условиях $a_0 = 1$.
Будем искать эпф описывающую эту последовательность, для этого домножаем все выражение на $$ \frac{x^{n+1}}{\left(n+1\right)!} $$ и суммируем: $$ \sum_{n=0}^{\infty} a_{n+1} \frac{x^{n+1}}{\left(n+1\right)!} = \sum_{n=0}^{\infty} a_n\left(n+1\right) \frac{x^{n+1}}{\left(n+1\right)!} = x\sum_{n=0}^{\infty} a_n \frac{x^n}{n!} $$
откуда получаем, что эпф $$ F\left(x\right) = \frac{a_0}{1 - x} = a_0 \sum_{n=0}^{\infty} n! \frac{x^n}{n!}$$
